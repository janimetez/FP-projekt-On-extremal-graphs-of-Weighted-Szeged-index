\documentclass[12pt, a4paper]{article}
\usepackage[utf8]{inputenc}
\usepackage[T1]{fontenc}
\usepackage[slovene]{babel}
\usepackage{amsmath}
\usepackage{eurosym}
\usepackage{hyperref}
\usepackage{graphicx}
\usepackage[top=2.5cm, bottom=2.5cm, left=3cm, right=2.5cm]{geometry}
\usepackage{indentfirst}
\usepackage{tikz}
\setlength{\parindent}{0.5cm}


\begin{document}

\begin{titlepage}
\begin{center}

\large
Univerza v Ljubljani\\
\normalsize
Fakulteta za matematiko in fiziko\\

\vspace{3 cm} 

\large
Bor Rotar, Jani Metež\\

\vspace{3 cm}
\LARGE
\textbf{O ekstremnih grafih v povezavi z obteženim Szegedovim indeksom}

\vfill

\large Ljubljana, 2020

\end{center}
\end{titlepage}

\newpage

\section[Definiranje problema]{Definiranje problema}

V projektni nalogi bova pokazala, da so grafi, ki minimizirajo obteženi Szgadov indeks (wSz), za 26 ali več vozlišč drevesa in jih pokazala. Hkrati bova skušala ugotoviti čim več lastnosti teh dreves.\\

Obteženi Szgedov indeks:
\begin{center}
 $$wSz(G) =\sum_{e=uv \in E(G)}[deg(u)+ deg(v) ]\cdot n_{u}(e)\cdot n_{v}(e)$$.
\end{center}
\medskip
Pripombe:
\begin{enumerate}
\item $deg(u)$ je stopnja vozlišča
\item $n_{u}(e)$ je moč množice vseh vozlišč, ki so bližje $u$ kot pa $v$ \\ (vključno z $u$ in $v$)
\item Obteženi Szgadov indeks je definiran za enostavne grafe
\end{enumerate}
\medskip

\medskip
Potek dela: \\

V $Sage$-u oz. $Cocalc$-u bo potrebno definirati $wSz$ in ugotoviti čim bolj enostaven način za generiranje grafov, ki minimizirajo ta indeks. Ko bo to storjeno, bo potrebno le še opaziti čim več možnih lastnosti teh grafov in od katerega števila vozlišč naprej veljajo. Nekatere lastnosti že poznamo iz vira na katergea se navezuje projekt.

\medskip


\newpage
\section[Algoritmi]{Algoritmi}

\subsection[Obteženi Szgadov indeks]{Obteženi Szgadov indeks}

Obteženi Szgadov indeks bomo označili z $wSz$. Definiran bo tako, da se bo zapeljal čez vsako povezavo.
 
\begin{verbatim}
def wSz(M):
    indeks = []
    d = M.distance_all_pairs()
    for u,v in M.edges(labels = false):
        blizu_u = 0
        for a in M.vertices():
            if d[a][u] < d[a][v]:
                blizu_u += 1
            blizu_v = order(M)- blizu_u
        indeks += [(M.degree(u) + M.degree(v)) * blizu_u * blizu_v]
    return sum(indeks)
\end{verbatim}

\subsection[Spreminjanje grafa]{Spreminjanje grafa}

Algoritem vzame nek povezan graf in mu dodaja ali odstranjuje povezave, pri tem pazi na to, da graf ostaja povezan (wSz je definiran za enostavne torej povezane grafe). 

\begin{verbatim}
from sage.graphs.connectivity import is_connected
def spremeni_graf(G):
    H = Graph(G)
    if random() < 0.5:
        i = 0
        while True:
            H.delete_edge(H.random_edge())
            if is_connected(H):
                H
                break
            else:
                H = Graph(G)
                i = i + 1
                True
            if i > 15: 
                H.add_edge(H.complement().random_edge()) 
                break
    return H
\end{verbatim}

Dodamo še preprosto funkcijo, ki grafu doda novo vozlišče in ga poveže z prvim vozliščem grafa.

\begin{verbatim}
def novo_vozlisce(G):
    H = Graph(G)
    novo_vzl = order(H)
    H.add_edge((0,novo_vzl,None))
    return H
\end{verbatim}

\newpage

\subsection[Algoritem za minimiziranje wSz]{Algoritem za minimiziranje wSz}

Ko pogledamo grafe z minimalnim $wSz$ do 25 vozlišč opazimo, da so si podobni. Sklepamo: če grafu na $n$ vozliščih, ki že ima minimalni $wSz$, dodamo novo vozlišče s povezavo, bomo lažje prišli do grafa z minimalnim $wSz$, kot pa če to iščemo na čisto novem grafu.\\
\\
Torej, da najdemo graf na $n$ vozliščih z $min(wSz)$ bomo grafu na $(n-1)$ vozliščih z že minimiziranim $wSz$ dodali vozlišle in povezavo in ga spreminjali ter te spremembe obdržali, če je $wSz$ novega grafa manjši. Ko $wSz$ ne bomo mogli več zmanjšati, algoritem oz. ponovitve algoritma zaključimo.

\begin{verbatim}
def min_wSz(H,koraki):
    k = 0
    primerjajH = H
    HwSz = wSz(H)
    while k < koraki:
        k = k + 1
        H = spremeni_graf(H)
        MwSz = wSz(H)
        if MwSz < HwSz:
            primerjajH = H
    return primerjajH
\end{verbatim}

V algoritem pa lahko tudi vstavimo graf za katerega sklepamo, da ima minimalni $wSz$, in če ne najde boljšega grafa lahko sklepamo, da ima ta graf že minimalni $wSz$.

\newpage
\begin{table}[!h]
\begin{tabular}{|c|c|c|c|c|c|c|c|}
\hline
n&&&&&&&\\\hline
&&&&&&&\\\hline
&&&&&&&\\\hline
&&&&&&&\\\hline
\end{tabular}
\end{table}


\newpage
\section[Viri]{Viri}

\begin{thebibliography}{9}
\bibitem{latexcompanion} 
Jan Boka, Boris Furtulab, Nikola Jedličkova in Riste Škrekovski
\textit{On Extremal Graphs of Weighted Szeged Index} 
https://arxiv.org/abs/1901.04764, 15 Jan 2019.
 

\end{thebibliography}


\end{document}